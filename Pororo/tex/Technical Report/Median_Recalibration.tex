%%%%%%%%%%%%%%%%%%%%%%%%%%%%%%%%%%%%%%%%%
% University/School Laboratory Report
% LaTeX Template
% Version 3.1 (25/3/14)
%
% This template has been downloaded from:
% http://www.LaTeXTemplates.com
%
% Original author:
% Linux and Unix Users Group at Virginia Tech Wiki 
% (https://vtluug.org/wiki/Example_LaTeX_chem_lab_report)
%
% License:
% CC BY-NC-SA 3.0 (http://creativecommons.org/licenses/by-nc-sa/3.0/)
%
%%%%%%%%%%%%%%%%%%%%%%%%%%%%%%%%%%%%%%%%%
% Technical Report on the recalibration of Tobii Glasses Eye Tracker.
%
% Author:
% Jin-Hwa Kim (jhkim@bi.snu.ac.kr)        07/10/2014
%
%%%%%%%%%%%%%%%%%%%%%%%%%%%%%%%%%%%%%%%%%
%----------------------------------------------------------------------------------------
% PACKAGES AND DOCUMENT CONFIGURATIONS
%----------------------------------------------------------------------------------------

\documentclass{article}

\usepackage{graphicx} % Required for the inclusion of images
\usepackage{natbib} % Required to change bibliography style to APA

\setlength\parindent{0pt} % Removes all indentation from paragraphs

\renewcommand{\labelenumi}{\alph{enumi}.} % Make numbering in the enumerate environment by letter rather than number (e.g. section 6)

%\usepackage{times} % Uncomment to use the Times New Roman font

%----------------------------------------------------------------------------------------
% DOCUMENT INFORMATION
%----------------------------------------------------------------------------------------

\title{Technical Report on the Median Recalibration \\
for Eye Tracking Data on Watching Video.} % Title

\author{Jin-Hwa \textsc{Kim}} % Author name

\date{\today} % Date for the report

\begin{document}

\maketitle % Insert the title, author and date

\begin{center}
\begin{tabular}{l r}
Date Performed: & June 24, 2014 \\ % Date the experiment was performed
Partners: & Eun-Sol Kim \\ % Partner names
& Kyoung-Woon On \\
Instructor: & Professor Byoung-Tak Zhang % Instructor/supervisor
\end{tabular}
\end{center}

% If you wish to include an abstract, uncomment the lines below
% \begin{abstract}
% Abstract text
% \end{abstract}

%----------------------------------------------------------------------------------------
% SECTION 1
%----------------------------------------------------------------------------------------

\section{Objective}

To determine that the recalibration using the difference vector between the center of the screen and the median of fixation points is effective to minimize the error of the calibration. All gaze points are adjusted by adding the vector which is the difference vector $v$.

\begin{center}${p}' \leftarrow p + v, $\\
$v \leftarrow p_{center} - p_{median}$\end{center}

% If you have more than one objective, uncomment the below:
%\begin{description}
%\item[First Objective] \hfill \\
%Objective 1 text
%\item[Second Objective] \hfill \\
%Objective 2 text
%\end{description}

\subsection{Definitions}
\label{definitions}
\begin{description}

\item[Calibration]
The calibration procedure manually moved the calibration point for the acquirement of the accurate gaze points. This internal calibration point is only valid while the glasses are not removed from the participant's head. Accuracy depends on the variety factors, which are right distance, exact points and light conditions, during the calibration procedure.

\item[Recalibration]
Tobii Glasses eye tracker, which is used by our experiment, provides the post calibration which can be conducted the calibration procedure after the eye movement recording. This is called as \textit{Post calibration} or \textit{recalibration}. However, in this technical report, we define the \textit{recalibration} as the procedure for the refinement of the previous device calibration.

\item[Gaze Point]
Gaze point means literally the location that the participant gazes.

\item[Fixation Point]
Fixation point means the location that the participant fixes his or her gaze point for some time. Because the gaze points are always changing subtly even in the voluntary fixation state, the definition of the fixation is defined by the fixation filtering method. We used the system default fixation filter, I-VT fixation filter \citep{Salvucci2000}.

\end{description} 
 
%----------------------------------------------------------------------------------------
% SECTION 2
%----------------------------------------------------------------------------------------

\section{Experiments}

To verify the improvement of the median recalibration method, we used the gaze accuracy test in advance of the watching the video. The gaze accuracy test will show us how much accurate before and after the recalibration.

\subsection{Gaze Accuracy Test}
For the measurement of the accuracy of the gaze point, we prepared the test video which has the moving dot on the lattice background. Participants are instructed to patiently fixate on the dot while it is stationary. You can check the video on YouTube at \citet{GAT}.

\subsection{Material}
We used the portion of episodes in the kids video, \textit{Pororo Season 3}, as the material to get the median point of fixations. The playing time is approximately 30 minutes, which yields about 3,000 to 6,000 fixations. Though the gaze accuracy test is given to only two participants among 18 participants, we will observe the median recalibration results on all 18 participants, who watched the video.

%----------------------------------------------------------------------------------------
% SECTION 3
%----------------------------------------------------------------------------------------

\section{Results and Conclusions}

The atomic weight of magnesium is concluded to be \SI{24}{\gram\per\mol}, as determined by the stoichiometry of its chemical combination with oxygen. This result is in agreement with the accepted value.

\begin{figure}[h]
\begin{center}
\includegraphics[width=0.65\textwidth]{placeholder} % Include the image placeholder.png
\caption{Figure caption.}
\end{center}
\end{figure}

%----------------------------------------------------------------------------------------
% SECTION 5
%----------------------------------------------------------------------------------------

\section{Discussion of Experimental Uncertainty}

The accepted value (periodic table) is \SI{24.3}{\gram\per\mole} \cite{Smith:2012qr}. The percentage discrepancy between the accepted value and the result obtained here is 1.3\%. Because only a single measurement was made, it is not possible to calculate an estimated standard deviation.

The most obvious source of experimental uncertainty is the limited precision of the balance. Other potential sources of experimental uncertainty are: the reaction might not be complete; if not enough time was allowed for total oxidation, less than complete oxidation of the magnesium might have, in part, reacted with nitrogen in the air (incorrect reaction); the magnesium oxide might have absorbed water from the air, and thus weigh ``too much." Because the result obtained is close to the accepted value it is possible that some of these experimental uncertainties have fortuitously cancelled one another.

%----------------------------------------------------------------------------------------
% SECTION 6
%----------------------------------------------------------------------------------------

\section{Answers to Definitions}

\begin{enumerate}
\begin{item}
The \emph{atomic weight of an element} is the relative weight of one of its atoms compared to C-12 with a weight of 12.0000000$\ldots$, hydrogen with a weight of 1.008, to oxygen with a weight of 16.00. Atomic weight is also the average weight of all the atoms of that element as they occur in nature.
\end{item}
\begin{item}
The \emph{units of atomic weight} are two-fold, with an identical numerical value. They are g/mole of atoms (or just g/mol) or amu/atom.
\end{item}
\begin{item}
\emph{Percentage discrepancy} between an accepted (literature) value and an experimental value is
\begin{equation*}
\frac{\mathrm{experimental\;result} - \mathrm{accepted\;result}}{\mathrm{accepted\;result}}
\end{equation*}
\end{item}
\end{enumerate}

%----------------------------------------------------------------------------------------
% BIBLIOGRAPHY
%----------------------------------------------------------------------------------------

\bibliographystyle{apalike}

\bibliography{Median_Recalibration}

%----------------------------------------------------------------------------------------


\end{document}